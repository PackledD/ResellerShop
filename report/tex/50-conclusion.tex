\chapter*{\hfill{\centering ЗАКЛЮЧЕНИЕ}\hfill}
\addcontentsline{toc}{chapter}{ЗАКЛЮЧЕНИЕ}

В ходе выполнения курсовой работы был проведен анализ алгоритмов построения трехмерного изображения сцены, генерации псевдослучайных чисел и сглаживания. В результате анализа были выбраны полигональная модель, алгоритм $Z$-буфера, модель Ламберта, метод Гуро, вихрь Мерсенна и фильтры: медианный фильтр, фильтр максимумов, фильтр Гаусса.

На основе выбранных алгоритмов было разработано ПО, которое должно позволять генерировать поверхность с использованием шума Перлина по заданным параметрам.

Для разработанного ПО был разработан интерфейс.
Разработанное ПО и выбранные алгоритмы были реализованны.

Проведены исследования реализованного ПО: выяснено влияние частоты и алгоритмов сглаживания на генерируемую поверхность, а также выявлена зависимость времени генерации от количества октав. Повышение частоты генерации повышает качество поверхности, алгоритмы сглаживания позволяют изменять форму рельефа поверхности, а с увеличением числа октав время генерации поверхности возрастает.

В дальнейшем, работа может быть расширена: например, могут быть добавлены для сравнения другие методы генерации поверхностей (Diamond-Square, Шум Вороного) или другие фильтры (фильтр эрозии, усредняющий фильтр).