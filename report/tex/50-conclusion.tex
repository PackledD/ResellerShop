\chapter*{\hfill{\centering ЗАКЛЮЧЕНИЕ}\hfill}
\addcontentsline{toc}{chapter}{ЗАКЛЮЧЕНИЕ}

В ходе выполнения курсовой работы была проанализирована предметная область и существующие решения, формализованы пользователи и сущности.

На основе этой формализации были сроектированы БД и ПО, описаны сущности и ограничения целостности.

Разработанные БД и ПО были реализованы.
Был реализован также Web-интерфейс для ПО.

Было проведено исследование зависимости времени выполнения запроса от размера таблицы в БД в двух случаях: без использования дополнительного индекса и с его использованием.
По результатам исследования выявлено следующее:
\begin{itemize}
	\item[---] с увеличением количества записей в таблице увеличивается время поиска;
	\item[---] при отсутствии дополнительного индекса затрачиваемое время возрастает быстрее, чем при его наличии;
	\item[---] при наличии дополнительного индекса время запроса по полю email все равно больше, чем по полю id; возможно, это объясняется тем, что сравнение числовых значений работает быстрее, чем текстовых;
	\item[---] при количестве записей в пределах 100000 индекс ускоряет запрос примерно на 5--10\%.
\end{itemize}

В дальнейшем, работа может быть расширена: например, могут быть добавлены новые сценарии использования (поиск товара по названию, сортировка по цене и т.д.) и поддержка других СУБД.