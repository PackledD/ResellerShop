\chapter{Аналитический раздел}

В данном разделе анализируется предметная область и существующие решения, рассматривается классификация СУБД.

\section{Анализ предметной области}
Реселлинг (от англ. resale, также реселл/ресейл) --- это процесс покупки товаров с целью их последующей перепродажи, как правило, с прибылью.

Актуальность реселлинга обусловлена несколькими факторами.
\begin{enumerate}
	\item Возможность продать товар компании, которая будет его реализовывать сама (удобство).
	\item Возможность дать вторую жизнь в других руках неиспользуемому электронному товару, а не просто выкинуть его (экологичность).
	\item Наличие множества складов и возможность использования любого из них для реализации товара.
\end{enumerate}

\section{Анализ существующих решений}
Среди компаний, занимающихся реселлингом электронных товаров на территории РФ, можно выделить SYRO, Мосгорломбард и ЭлектроМаркет.
В таблице \ref{table1} приведен сравнительный анализ этих компаний по следующим критериям:
\begin{itemize}
	\item[---] наличие складов;
	\item[---] наличие ремонтного сервиса;
	\item[---] ассортимент товаров.
\end{itemize}

\begin{table}[]
	\captionsetup{singlelinecheck=off}
	\caption{\label{table1}Сравнение компаний-реселлеров электроники}
	\begin{tabular}{|r|c|c|r|}
		\hline
		\multicolumn{1}{|c|}{Название} & \begin{tabular}[c]{@{}c@{}}Наличие\\ склада\end{tabular} & \begin{tabular}[c]{@{}c@{}}Наличие ремонт-\\ ного сервиса\end{tabular} & \multicolumn{1}{c|}{\begin{tabular}[c]{@{}c@{}}Ассортимент\\ товаров\end{tabular}} \\ \hline
		SYRO                           & +                                                        & +                                                                      & Только электротовары                                                                    \\ \hline
		Мосгорломбард                  & +                                                        & -                                                                      & Все товары                                                                              \\ \hline
		ЭлектроМаркет                  & -                                                        & -                                                                      & Только электротовары                                                                    \\ \hline
	\end{tabular}
\end{table}

\section{Классификация СУБД}
СУБД классифицируются по различным критериям, в зависимости от их архитектуры, способа доступа к БД, модели хранения данных и модели обработки данных.

\subsection{Модель хранения данных}
По модели хранения данных выделяют несколько типов СУБД:
\begin{itemize}
	\item[---] иерархические;
	\item[---] сетевые;
	\item[---] реляционные;
	\item[---] объектно-ориентированные.
\end{itemize}

Иерархические СУБД организовывают данные в виде иерархии, где каждый узел имеет родителя и может иметь несколько дочерних узлов.

Сетевые СУБД хранят данные в виде графа, где сущность может быть связана с несколькими другими сущностями.

Реляционные СУБД хранят данные в виде таблиц, которые состоят из строк и столбцов.
Реляционные базы и используют структурированный язык запросов (SQL) для доступа к данным.

Объектно-ориентированные СУБД организовывают хранение данных в виде объектов и классов, что позволяет эффективно работать с комплексными данными и их отношениями.

\subsection{Модель обработки данных}
По модели обработки данных, СУБД разделяются на OLTP и OLAP.

OLTP (Online Transaction Processing) базы данных предназначены для обработки оперативных транзакций (оформление заказов, бронирование и т.д.) в реальном времени.
OLTP-модель как правило используется для быстрой записи, обновления и удаления данных.
Главная цель OLTP --- обеспечить быструю и надежную обработку операций.

OLAP (Online Analytical Processing) базы данных предназначены для анализа и отчетности. Эффективно работают с крупными объемами данных.
OLAP-модель обычно используется для агрегации данных, создания сводных таблиц, проведения аналитических запросов и прогнозирования.

\subsection{Архитектура}
По архитектуре организации хранения данных, СУБД разделяются на локальные и распределенные.

Локальная СУБД устанавливается и работает на одном компьютере (сервере).
Она хранит и обрабатывает данные только на нем.
Примерами локальных СУБД являются SQLite, Microsoft Access и PostgreSQL (если установлена на одном сервере).

Распределенные СУБД распределяют хранение и обработку данных между несколькими устройствами (серверами).
Распределенные СУБД обычно имеют более высокую производительность и масштабируемость, поскольку распределение данных по разным узлам сети уменьшает нагрузку на отдельные устройства.
Примерами распределенных СУБД являются MongoDB и Cassandra.

\subsection{Способ доступа к БД}
По способу доступа к БД выделяют следующие виды СУБД:
\newpage
\begin{itemize}
	\item[---] файл-серверные (Microsoft Access);
	\item[---] клиент-серверные (PostgreSQL, Oracle, MS SQL Server);
	\item[---] встраиваемые (Redis, SQLite);
	\item[---] сервисно-ориентированные;
	\item[---] другие.
\end{itemize}

Файл-серверные СУБД используются для хранения данных в файлах на сервере.
В этом случае клиенты обращаются к серверу для доступа ко всем данным сразу.
При такой организации доступа, каждый пользователь хранит на своем компьютере локальную копию БД, а запросы выполняются локально.

Клиент-серверные СУБД представляют собой систему, в которой клиентские компьютеры соединены с сервером, который управляет хранением и обработкой данных.
Клиенты обращаются к серверу для доступа только к нужным данным.
Запросы обрабатываются на сервере.

Встраиваемые СУБД представляют собой хранилище, интегрированное в ПО или устройство.
Они работают непосредственно в рамках программы и не требуют отдельной установки  или запуска сервера.

Сервисно-ориентированные СУБД используются для хранения данных, предоставляемых через веб-сервисы.
В этом случае доступ к данным и обработка запросов для клиентов осуществляется через API.

%\section{Выбор СУБД}
%Была выбрана СУБД PostgreSQL. Это реляционная клиент-серверная СУБД с поддержкой языка SQL.

\section{Вывод из аналитического раздела}
В данном разделе была проанализирована предметная область и существующие решения, рассмотрена классификация СУБД.