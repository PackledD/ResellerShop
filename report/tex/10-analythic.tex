\chapter{Аналитический раздел}

В данном разделе анализируются существующие решения, рассматривается классификация СУБД, а также приводятся ER-диаграмма сущностей, формализация пользователей и диаграмма вариантов использования.

\section{Анализ существующих решений}
Среди компаний, занимающихся реселлингом товаров на территории РФ, можно выделить SYRO \cite{src_site1}, OZON \cite{src_site2}, Яндекс.Маркет \cite{src_site3}, Second Friend Store \cite{src_site4}.
В таблице \ref{table1} приведен сравнительный анализ этих компаний по следующим критериям:
\begin{itemize}
	\item[---] наличие собственных складов;
	\item[---] наличие ремонтного сервиса;
	\item[---] ассортимент товаров;
	\item[---] наличие комиссии.
\end{itemize}

\begin{table}[!h]
	\captionsetup{singlelinecheck=off}
	\caption{\label{table1}Сравнение компаний-реселлеров}
	\begin{tabular}{|r|c|c|c|c|}
		\hline
		\multicolumn{1}{|c|}{Название} & \begin{tabular}[c]{@{}c@{}}Наличие\\ складов\end{tabular} & \begin{tabular}[c]{@{}c@{}}Наличие\\ремонтного \\сервиса\end{tabular} & \multicolumn{1}{c|}{\begin{tabular}[c]{@{}c@{}}Ассортимент\\ товаров\end{tabular}} & \multicolumn{1}{c|}{\begin{tabular}[c]{@{}c@{}}Наличие\\ комиссии\end{tabular}} \\ \hline
		SYRO                           & +                                                        & +                                                                      & \multicolumn{1}{|c|}{\begin{tabular}[c]{@{}c@{}}Только\\электротовары\end{tabular}}  & +                                                                  \\ \hline
		OZON                  & +                                                        & -                                                                      & Все товары & +                                                                              \\ \hline
		Яндекс.Маркет                  & +                                                        & -                                                                      & Все товары & +                                                                  \\ \hline
		\multicolumn{1}{|c|}{\begin{tabular}[c]{@{}c@{}}Second Friend\\ Store\end{tabular}}                   & -                                                        & -                                                                      & \multicolumn{1}{|c|}{\begin{tabular}[c]{@{}c@{}}Только\\одежда\end{tabular}}     & -\\                                                             \hline
	\end{tabular}
\end{table}

\section{Классификация СУБД}
СУБД классифицируются по различным критериям, в зависимости от их архитектуры, способа доступа к БД, модели хранения данных и модели обработки данных \cite{src_db}.

\subsection{Модель хранения данных}
По модели хранения данных выделяют несколько типов СУБД \cite{src_db}:
\begin{itemize}
	\item[---] иерархические;
	\item[---] сетевые;
	\item[---] реляционные;
	\item[---] объектно-ориентированные.
\end{itemize}

Иерархические СУБД организовывают данные в виде иерархии, где каждый узел имеет родителя и может иметь несколько дочерних узлов.

Сетевые СУБД хранят данные в виде графа, где сущность может быть связана с несколькими другими сущностями.

Реляционные СУБД хранят данные в виде таблиц, которые состоят из строк и столбцов.
Реляционные базы и используют структурированный язык запросов (SQL) для доступа к данным.

Объектно-ориентированные СУБД организовывают хранение данных в виде объектов и классов, что позволяет эффективно работать с комплексными данными и их отношениями.

\subsection{Модель обработки данных}
По модели обработки данных, СУБД разделяются на OLTP и OLAP~\cite{src_olap_oltp}.

OLTP (Online Transaction Processing) базы данных предназначены для обработки оперативных транзакций (оформление заказов, бронирование и т.д.) в реальном времени.
OLTP-модель как правило используется для быстрой записи, обновления и удаления данных.
Главная цель OLTP --- обеспечить быструю и надежную обработку операций.

OLAP (Online Analytical Processing) базы данных предназначены для анализа и отчетности. Эффективно работают с крупными объемами данных.
OLAP-модель обычно используется для агрегации данных, создания сводных таблиц, проведения аналитических запросов и прогнозирования.

\subsection{Архитектура}
По архитектуре организации хранения данных, СУБД разделяются на локальные и распределенные \cite{src_db}.

Локальная СУБД устанавливается и работает на одном компьютере (сервере).
Она хранит и обрабатывает данные только на нем.
Примерами локальных СУБД являются SQLite, Microsoft Access и PostgreSQL (если установлена на одном сервере).

Распределенные СУБД распределяют хранение и обработку данных между несколькими устройствами (серверами).
Распределенные СУБД обычно имеют более высокую производительность и масштабируемость, поскольку распределение данных по разным узлам сети уменьшает нагрузку на отдельные устройства.
Примерами распределенных СУБД являются MongoDB и Cassandra.

\subsection{Способ доступа к БД}
По способу доступа к БД выделяют следующие виды СУБД \cite{src_db}:
\begin{itemize}
	\item[---] файл-серверные (Microsoft Access);
	\item[---] клиент-серверные (PostgreSQL, Oracle, MS SQL Server);
	\item[---] встраиваемые (Redis, SQLite);
	\item[---] сервисно-ориентированные;
	\item[---] другие.
\end{itemize}

Файл-серверные СУБД используются для хранения данных в файлах на сервере.
В этом случае клиенты обращаются к серверу для доступа ко всем данным сразу.
При такой организации доступа, каждый пользователь хранит на своем компьютере локальную копию БД, а запросы выполняются локально.

Клиент-серверные СУБД представляют собой систему, в которой клиентские компьютеры соединены с сервером, который управляет хранением и обработкой данных.
Клиенты обращаются к серверу для доступа только к нужным данным.
Запросы обрабатываются на сервере.

Встраиваемые СУБД представляют собой хранилище, интегрированное в ПО или устройство.
Они работают непосредственно в рамках программы и не требуют отдельной установки  или запуска сервера.

Сервисно-ориентированные СУБД используются для хранения данных, предоставляемых через веб-сервисы.
В этом случае доступ к данным и обработка запросов для клиентов осуществляется через API.

\section{Формулирование требований к БД и ПО}
ПО должно представлять собой приложение с Web-интерфейсом.

Разрабатываемое ПО должно предоставлять пользователям доступ к данным о товарах, их персональных данных и данных их фирм.
Должна быть реализована возможность выбора всех товаров и товаров по категориям.
Должна быть реализована система регистрации и авторизации.
Представители фирм должны иметь возможность заключать и подписывать контракты.

Администратор должен иметь возможность управлять сущностями системы (фирмами, пользователями, складами, товарами, производителями, категориями).

Аудитор должен иметь возможность просматривать все контракты и сопутствующие им документы.
\newpage
\section{Ролевая модель}
Выделены следующие роли:
\begin{enumerate}
	\item Менеджер --- менеджер фирмы; может управлять товарами (добавлять, изменять и удалять), а также составлять контракты от лица своей фирмы.
	\item Руководитель --- руководитель фирмы; может управлять товарами (добавлять, изменять и удалять), а также подписывать контракты от лица своей фирмы.
	\item Администратор --- суперпользователь; может управлять (добавлять, изменять и удалять) всеми компонентами системы (фирмами, складами, товарами, производителями, категориями и пользователями).
	\item Аудитор --- может просматривать список всех контрактов и приложенные к ним документы.
\end{enumerate}
Каждая фирма при заключении контракта может выступать в роли как покупателем, так и продавца.

В соответствии с ролевой моделью, на рисунке \ref{use_case} приведена диаграмма прецедентов.
\begin{figure}[h!]
	\center{\includesvg[inkscapelatex=false,width=\textwidth]{use_case.svg}}
	\caption{Диаграмма прецедентов}
	\label{use_case}
\end{figure}

\newpage
\section{ER-модель}
На рисунке \ref{er_model} представлена ER-модель в нотации Чена.
\begin{figure}[h!]
	\center{\includesvg[inkscapelatex=false,width=0.95\textwidth]{er.svg}}
	\caption{ER-модель в нотации Чена}
	\label{er_model}
\end{figure}

\newpage
\section{Вывод из аналитического раздела}
В данном разделе были проанализированы существующие решения, рассмотрена классификация СУБД, а также были приведены ER-диаграмма сущностей, формализация пользователей и диаграмма вариантов использования.