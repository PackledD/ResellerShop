\chapter{Конструкторский раздел}
В данном разделе рассмотрено проектирование БД и ПО, а также приведено описание сущностей и ограничений целостности.

\section{Описание сущностей и ограничений целостности}
\begin{enumerate}
	\item Фирма имеет следующие атрибуты: 
	\begin{itemize}
		\item[---] ID, PK;
		\item[---] название;
		\item[---] телефон;
		\item[---] email;
		\item[---] физический адрес;
		\item[---] юридический адрес.
	\end{itemize}
	
	\item Пользователь имеет следующие атрибуты: 
	\begin{itemize}
		\item[---] ID, PK;
		\item[---] ФИО;
		\item[---] фирма, FK;
		\item[---] телефон;
		\item[---] email UNIQUE.
	\end{itemize}

	\item Товар имеет следующие атрибуты: 
	\begin{itemize}
		\item[---] ID, PK;
		\item[---] название;
		\item[---] категория, FK;
		\item[---] поставщик, FK;
		\item[---] стоимость;
		\item[---] производитель, FK.
	\end{itemize}
\newpage
	\item Категория товара имеет следующие атрибуты: 
	\begin{itemize}
		\item[---] ID, PK;
		\item[---] название.
	\end{itemize}
	
		\item Производитель имеет следующие атрибуты: 
	\begin{itemize}
		\item[---] ID, PK;
		\item[---] название.
	\end{itemize}
	
	\item Склад имеет следующие атрибуты: 
	\begin{itemize}
		\item[---] ID, PK;
		\item[---] адрес.
	\end{itemize}
	
	\item Контракт имеет следующие атрибуты: 
	\begin{itemize}
		\item[---] ID, PK;
		\item[---] ID фирмы, FK;
		\item[---] руководитель 1, может быть Null, FK;
		\item[---] руководитель 2, может быть Null, FK;
		\item[---] менеджер 1, может быть Null, FK;
		\item[---] менеджер 2, может быть Null, FK;
		\item[---] дата заключения;
		\item[---] дата истечения;
		\item[---] документ.
	\end{itemize}

	\item Товар на складе имеет следующие атрибуты: 
	\begin{itemize}
		\item[---] ID товара, FK;
		\item[---] ID склада, FK;
		\item[---] количество. 
	\end{itemize}
	\newpage
	\item Позиция контракта имеет следующие атрибуты: 
	\begin{itemize}
		\item[---] ID контракта, FK;
		\item[---] ID товара, FK;
		\item[---] ID склада, FK;
		\item[---] количество. 
	\end{itemize}

	\item Данные аутентификации имеют следующие атрибуты: 
	\begin{itemize}
		\item[---] ID пользователя, FK;
		\item[---] хэш пароля. 
	\end{itemize}
\end{enumerate}

Все атрибуты описываемых сущностей, кроме помеченных атрибутов контракта, не могут принимать значение Null.

Атрибуты, помеченные как PK, являются ключевыми. Атрибуты, помеченные как FK, являются внешними.

Атрибут пользователя email также является уникальным.

Диаграмма проектируемой БД представлена на рисунке \ref{db_diagram}.
\begin{figure}[h!]
	\center{\includesvg[inkscapelatex=false,width=\textwidth]{db_diagram.svg}}
	\caption{Диаграмма БД}
	\label{db_diagram}
\end{figure}

\section{Проектирование функции на стороне БД}
Проектируемая функция будет возвращать значение атрибута Id для добавляемого в таблицу объекта.
Схема проектируемой функции приведена на рисунке \ref{function}.
\begin{figure}[h!]
	\center{\includesvg[inkscapelatex=false,width=200px]{function.svg}}
	\caption{Проектируемая функция}
	\label{function}
\end{figure}

\section{Архитектура программы}
На рисунке \ref{up_level} приведена верхнеуровневая UML-диаграмма модулей.
\begin{figure}[h!]
	\center{\includesvg[inkscapelatex=false,width=\textwidth]{up_level.svg}}
	\caption{Верхнеуровневая UML-диаграмма модулей}
	\label{up_level}
\end{figure}
ПО состоит из трех компонентов: компонент бизнес-логики, компонент доступа к данным и компонент пользовательского интерфейса.
\newpage
На рисунке \ref{uml} приведена UML-диаграмма классов разрабатываемого ПО.
\begin{figure}[h!]
	\center{\includesvg[inkscapelatex=false,width=\textwidth]{uml.svg}}
	\caption{UML-диаграмма классов}
	\label{uml}
\end{figure}


\section{Вывод из конструкторского раздела}
В данном разделе было рассмотрено проектирование БД и ПО, описаны сущности и ограничения целостности.