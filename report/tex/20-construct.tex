\chapter{Конструкторский раздел}
В данном разделе рассмотрено проектирование ПО.

\section{Описание сущностей}
\begin{enumerate}
	\item Фирма имеет следующие атрибуты: 
	\begin{itemize}
		\item[---] ID;
		\item[---] название;
		\item[---] телефон;
		\item[---] email;
		\item[---] физический адрес;
		\item[---] юридический адрес.
	\end{itemize}
	
	\item Менеджер имеет следующие атрибуты: 
	\begin{itemize}
		\item[---] ID;
		\item[---] ФИО;
		\item[---] фирма;
		\item[---] телефон;
		\item[---] email.
	\end{itemize}
	
	\item Руководитель имеет следующие атрибуты: 
	\begin{itemize}
		\item[---] ID;
		\item[---] ФИО;
		\item[---] фирма;
		\item[---] телефон;
		\item[---] email.
	\end{itemize}
\newpage
	\item Товар имеет следующие атрибуты: 
	\begin{itemize}
		\item[---] ID;
		\item[---] название;
		\item[---] категория;
		\item[---] поставщик;
		\item[---] стоимость;
		\item[---] производитель.
	\end{itemize}
	
	\item Категория товара имеет следующие атрибуты: 
	\begin{itemize}
		\item[---] ID;
		\item[---] название.
	\end{itemize}
	
	\item Склад имеет следующие атрибуты: 
	\begin{itemize}
		\item[---] ID;
		\item[---] адрес.
	\end{itemize}
	
	\item Контракт имеет следующие атрибуты: 
	\begin{itemize}
		\item[---] ID;
		\item[---] ID фирмы;
		\item[---] руководитель 1;
		\item[---] руководитель 2;
		\item[---] менеджер 1;
		\item[---] менеджер 2;
		\item[---] дата заключения;
		\item[---] дата истечения;
		\item[---] документ.
	\end{itemize}
\newpage
	\item Товар на складе имеет следующие атрибуты: 
	\begin{itemize}
		\item[---] ID товара;
		\item[---] ID склада;
		\item[---] количество. 
	\end{itemize}
	
	\item Позиция контракта имеет следующие атрибуты: 
	\begin{itemize}
		\item[---] ID контракта;
		\item[---] ID товара;
		\item[---] ID склада;
		\item[---] количество. 
	\end{itemize}
\end{enumerate}

На рисунке \ref{er_model} представлена ER-модель в нотации Чена.
\begin{figure}[h!]
	\center{\includesvg[inkscapelatex=false,width=\textwidth]{er.svg}}
	\caption{ER-модель в нотации Чена}
	\label{er_model}
\end{figure}

\section{Ролевая модель}
Выделены следующие роли:
\begin{enumerate}
	\item Менеджер --- менеджер фирмы; может управлять товарами (добавлять, изменять и удалять), а также составлять контракты от лица своей фирмы.
	\item Руководитель --- руководитель фирмы; может управлять товарами (добавлять, изменять и удалять), а также подписывать контракты от лица своей фирмы.
	\item Администратор --- суперпользователь; может управлять (добавлять, изменять и удалять) всеми компонентами системы (фирмами, складами, товарами и пользователями).
\end{enumerate}

\clearpage
В соответствии с ролевой моделью, на рисунке \ref{use_case} приведена диаграмма прецедентов.
\begin{figure}[h!]
	\center{\includesvg[inkscapelatex=false,width=\textwidth]{use_case.svg}}
	\caption{Диаграмма прецедентов}
	\label{use_case}
\end{figure}

\clearpage
\section{Структура программы}
На рисунке \ref{up_level} приведена верхнеуровневая UML-диаграмма модулей.
\begin{figure}[h!]
	\center{\includesvg[inkscapelatex=false,width=\textwidth]{up_level.svg}}
	\caption{Верхнеуровневая UML-диаграмма модулей}
	\label{up_level}
\end{figure}
ПО состоит из трех компонентов: компонент бизнес-логики, компонент доступа к данным и компонент пользовательского интерфейса.

На рисунке \ref{uml} приведена UML-диаграмма классов разрабатываемого ПО.
\begin{figure}[h!]
	\center{\includesvg[inkscapelatex=false,width=\textwidth]{uml.svg}}
	\caption{UML-диаграмма классов}
	\label{uml}
\end{figure}


\section{Вывод из конструкторского раздела}
В данном разделе было рассмотрено проектирование ПО.