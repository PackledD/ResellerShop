\chapter*{\hfill{\centering ВВЕДЕНИЕ}\hfill}
\addcontentsline{toc}{chapter}{ВВЕДЕНИЕ}

Реселлинг (от англ. resale, также реселл/ресейл) --- это процесс покупки товаров с целью их последующей перепродажи, как правило, с прибылью \cite{src_resell}.

Актуальность реселлинга обусловлена несколькими факторами \cite{src_resell}.
\begin{enumerate}
	\item Возможность продать товар компании, которая будет его реализовывать сама (удобство).
	\item Возможность дать вторую жизнь в других руках неиспользуемым электронным товарам, а не просто выкинуть их (экологичность).
	\item Наличие множества складов и возможность использования любого из них для реализации товара.
\end{enumerate}

Цель курсовой работы --- разработка БД для хранения и обработки данных компании-реселлера электроники.

Для достижения поставленной цели требуется решить следующие задачи:
\begin{itemize}
	\item[---] проанализировать существующие решения;
	\item[---] сформулировать ограничения целостности и выделить роли;
	\item[---] описать сущности и ограничения целостности БД;
	\item[---] разработать БД и ПО для работы с ней;
	\item[---] реализовать ПО;
	\item[---] провести исследование характеристик разработанного ПО.
\end{itemize}