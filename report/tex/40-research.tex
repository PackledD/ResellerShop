\chapter{Исследовательский раздел}
В данном разделе проведено исследование влияния частоты генерации поверхности и применения алгоритмов сглаживания на результат ее генерации, а также зависимость времени генерации поверхности с использованием шума Перлина от количества октав.

\section{Вывод из исследовательского раздела}
В исследовательском разделе приведены результаты работы ПО и проведено исследование.
По результатам проведенного исследования выяснено:
\begin{itemize}
	\item[---]повышение частоты генерации увеличивает качество получаемой поверхности;
	\item[---]применение алгоритмов сглаживания значительно изменяет поверхность; фильтр максимумов создает <<плато>>, медианный фильтр сглаживает поверхность, создавая <<холмы>>, а фильтр Гаусса подчеркивает неровности;
	\item[---]с увеличением числа октав время генерации поверхности возрастает.
\end{itemize}
