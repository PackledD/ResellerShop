\chapter{Исследовательский раздел}
В данном разделе проведено исследование зависимости времени выполнения запроса от размера таблицы в БД в двух случаях: без использования дополнительного индекса и с его использованием.

\section{Описание исследования}
Исследование проводилось с использованием PgAdmin~---~графической оболочки для СУБД PostgreSQL.
В этом приложении есть встроенная возможность для измерения времени выполнения запросов \cite{src_pgadmin}.

Исследование производилось с использованием таблицы Users.
Поиск в таблице производился в трех вариантах: по полю Id, по полю email и по обоим полям сразу.

Предварительно таблица была заполнена случайными данными. Поиск выполнялся для последнего объекта.

В таблице \ref{table_res} представлены результаты исследования, время в мс.

\begin{table}[!h]
	\captionsetup{singlelinecheck=off}
	\caption{Результаты исследования\label{table_res}}
	\centering
	\begin{tblr}{
			cells = {r},
			row{1} = {c},
			row{2} = {c},
			cell{1}{1} = {r=2}{},
			cell{1}{2} = {c=3}{},
			cell{1}{5} = {c=3}{},
			vlines,
			hline{1,3-8} = {-}{},
			hline{2} = {2-7}{},
		}
		{К-во\\записей } & {Без дополнительного\\индекса} &       &     & {С дополнительным\\индексом} &       &     \\
		& Id                           & Email & Оба & Id                             & Email & Оба \\
		1000             &                          174 &   166 & 172 &                            162 &   156 & 153 \\
		5000             &                          171 &   185 & 177 &                            161 &   163 & 167 \\
		10000            &                          178 &   194 & 195 &                            170 &   177 & 177 \\
		50000            &                          184 &   199 & 205 &                            172 &   188 & 190 \\
		100000           &                          187 &   212 & 223 &                            176 &   190 & 193 
	\end{tblr}
\end{table}

\newpage
\section{Вывод из исследовательского раздела}
В данном разделе было проведено исследование зависимости времени выполнения запроса от размера таблицы в БД в двух случаях: без использования дополнительного индекса и с его использованием.
По результатам исследования выявлено следующее:
\begin{itemize}
	\item[---] с увеличением количества записей в таблице увеличивается время поиска;
	\item[---] при отсутствии дополнительного индекса затрачиваемое время возрастает быстрее, чем при его наличии;
	\item[---] при наличии дополнительного индекса время запроса по полю email все равно больше, чем по полю id; возможно, это объясняется тем, что сравнение числовых значений работает быстрее, чем текстовых;
	\item[---] при количестве записей в пределах 100000 индекс ускоряет запрос примерно на 5--10\%.
\end{itemize}
